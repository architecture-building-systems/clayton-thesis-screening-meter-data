Allocation of the primary use type of a building is often considered a trivial activity when analyzed from a smaller set of buildings. As the number of building being analyzed grows, so does the complexity of space use evaluation. The use of buildings changes over time and these changes are not always documented. In several of the case studies, this topic was discussed and highlighted as an issue with respect to benchmarking a building.\\
\\
Discriminatory features have already been visualized extensively in Section \ref{sec:featureextraction} and the differences between the primary use types are apparent in the overview heat maps of each feature. In this and the following sections, a quantification of the impact of each feature will be evaluated using a random forest model and its associated variable importance methods. Figure \ref{fig:use_classification} is the first such example of the output results of the classification model in predicting the building's primary use type using the temporal features created in this study. The model was built using the scikit-learn Python library with the number of estimators set to 100 and the minimum samples per leaf set to 2 \footnote{http://scikit-learn.org/}. The overall general accuracy of the model is 67.8\% as compared to a baseline model of 22.2\%. The baseline model using a stratified strategy in which categories are chosen randomly based on the percentage of each class occurring in the training set.