\section{Unsupervised Feature Extraction from Building and Urban Temporal Data}
\label{sec:litreview}

Researchers from several domains have developed methods of extracting insight from raw, unlabeled data from the built environment. Often these methods fall into the category of unsupervised statistical learning. Methods from this sub-domain of machine learning are advantageous due to their ability to characterize measured or simulated performance data quickly with less analyst intervention, metadata, and ground truth labeled data. 

\subsubsection{Load Profiling}
Load profiling is the process of grouping temporal subsequences of measured energy data for the purpose of characterizing the typical behavior of an individual customer. It involves time-series clustering and feature extraction. Chicco et al. \cite{chicco_support_2009} provides an original example in our review of this process using support vector machine clustering. Gullo et al. \cite{gullo_low-voltage_2009} and R\"as\"anen et al. \cite{rasanen_feature-based_2009} took the process further by introducing a framework of various clustering procedures that were implemented on case studies. Ramos et al. \cite{ramos_typical_2012}, Iglesias et al. \cite{iglesias_analysis_2013} and Panapakidis et al. \cite{panapakidis_evaluation_2015} tested various conventional and new clustering methods and similarity metrics in order to determine those most applicable to electrical load profiling. Chicco et al. \cite{chicco_electrical_2013} explored new clustering techniques based on ant colony grouping while Pan et al. \cite{pan_kernel-based_2015} discovered the use of kernal PCA for the same purpose. Several groups of reseachers such as Lavin and Klabjan \cite{lavin_clustering_2014} and Green et al. \cite{green_divide_2014} have found effective use in using the basic K-Means clustering algorithm for load profiling. Shahzadeh et al. \cite{shahzadeh_improving_2015} discussed the use of profiling as applied to forecast accuracy of temporal data. Two studies diverge from the standard profile development using clustering paradigm. The first is by De Silva et al. \cite{de_silva_data_2011} who uses Incremental Summarization and Pattern Characterization (ISPC) instead of clustering to find load profiles. The other is the visual analytics-based approach of creating a smart meter analytics dashboard by Nezhad et al. \cite{jarrah_nezhad_smartd:_2014} to create and inspect typical load profiles.

\subsection{Customer Classification}
Automated account classification is the next sub-category that utilizes unsupervised learning technques within the smart meter domain. These methods often employ load profile clustering as a first step but differentiate themselves in using those features to classify accounts, or buildings, that fit within various categories. Therefore, account classification is a type of manual semi-supervised analysis utilizing load profiling as a basis. The study by Figueiredo et al. \cite{figueiredo_electric_2005} harnessed K-Means and a labeled sample from accounts in Portugal to showcase this concept. Verdu et al. \cite{verdu_classification_2006} and R\"as\"anen et al. \cite{rasanen_reducing_2008} applied self-organizing maps (SOM) to accomplish a similar study that classifies accounts according to applicability of several demand response scenarios. Vale et al. \cite{vale_data_2009} gives an overview of a general data mining framework focused on characterizing customers. Florita et al. \cite{florita_classification_2012} diverge from the use of measured data by creating a massive amount of simulation data of load profiles to quantify energy storage applications for the power grid. Fagiani et al. \cite{fagiani_novelty_2015} use Markov Model novelty detection to automatically classify customers who potentially have leakage or waste issues. \c Cakmak et al. \cite{cakmak_new_2014} and Liu et al. \cite{liu_smas:_2015} test new visual analytics techniques within more holistic analysis framework for analyzing customers. Borgeson used various clustering and occupancy detection techniques to analyze a large AMI data set from California \cite{borgeson_targeted_2013}. Bidoki et al. tested various clustering techniques to evaluate applicability for customer classification \cite{bidoki_evaluating_2010}. A recent study in Korea develops a new clustering technique in segmenting customers to analyze demand response incentives \cite{jang_variability_2016}.

\subsection{Disaggregation}
The last area of smart meter data analysis is the field of meter disaggregation. Disaggregation attempts decompose a measurement signal from a high level reading to the individual loads being measured. This domain is well-researched from a supervised model perspective but recent attempts at unsupervised, pattern-based disaggregation were developed to facilitate implementation on unlabeled smart meter data. Shao et al. \cite{shao_temporal_2013} use Dirichlet Process Gaussian Mixture Models to find and disaggregate patterns in sub-hourly meter data. Reinhardt and Koessler \cite{reinhardt_powersax:_2014} use a version of symbolic aggregate approximation (SAX) to extract and identify disaggregated patterns for the purpose of prediction. These studies are also unique in that few of the disaggregation studies focus on commercial buildings as opposed to residential buildings.

\subsection{Portfolio Characterization}
Publications that address the characterization of a portfolio of buildings include unsupervised techniques meant to evaluate and visualize the range of behaviors and performance of the group. A majority of the techniques utilized are either clustering or visual analytics that provide a model of exploratory analysis that enable further steps. Seem \cite{seem_pattern_2005} produced an influential study that extracts days of the week with similar consumption profiles. Further clustering work was completed by An et al. \cite{an_estimation_2012} to estimate thermal parameters of a portfolio of buildings. Lam et al. \cite{lam_principal_2008} used Principal Component Analysis to extract information about a group of office buildings. Approaches focused on visual analytics and dashboards were completed by Agarwal et al. \cite{agarwal_energy_2009}, Lehrer \cite{lehrer_research_2009}, and Lehrer and Vasudev \cite{lehrer_visualizing_2011}. Granderson et al. \cite{granderson_building_2010} completed a case study-based evaluation of energy information systems, in which some methods combine some unsupervised approaches with visualization. Diong et al. \cite{diong_establishing_2015} completed a case study as well focused on a specific energy information system implementation. Mor\'an et al. \cite{moran_analysis_2013} and Georgescu and Mezic \cite{georgescu_site-level_2014} developed hybrid methods that employed visual continous maps and Koopman Operator methods respectively to visualize portfolio consumption. Miller et al. \cite{miller_forensically_2015,miller_automated_2015} completed two studies focused on the use of screening techniques to automatically extract diurnal patterns from performance data and use those patterns to characterize the consumption of a portfolio of buildings. Yarbrogh et al. used visual analytics techniques to analyze peak demand on a university campus \cite{yarbrough_visualizing_2015}.

\subsection{Portfolio Classification}
The concept of classifying buildings within a portfolio supplements the characterization techniques by assigning individual buildings to subgroups of relative performance for the purpose of benchmarking or decision-making. Santamouris et al. \cite{santamouris_using_2007} produced a report using clustering and classification to assign schools in Greece to subgroups of similar performance. Nikolaou et al. \cite{nikolaou_application_2012} and Pieri et al. \cite{pieri_identifying_2015} further extended this type of work to office buildings and hotels. Heidarinejad et al. \cite{heidarinejad_cluster_2014} released an analysis of clustered simulation data to classify LEED-certified office buildings. Ploennigs et al. \cite{ploennigs_e2-diagnoser:_2014} created a platform for monitoring, diagnosing and classifying buildings and operational behavior within a portfolio to quickly visualizing the outputs.

\subsection{Portfolio Targeting}
Targeting is a concept that builds upon characterization and classification to identify specific buildings or measures to be implemented in a portfolio to improve performance. These publications are differentiated in that specific measures are identified in the analysis. Sedano et al. \cite{sedano_improving_2009} uses Cooperative Maximum-Likelihood amongst other techniques to evaluate the thermal insulation performance of buildings. Gaitani et al. \cite{gaitani_using_2010} used PCA and clustering to target heating efficiency in school buildings. Bellala et al. \cite{bellala_towards_2011} used various methods to find lighting energy savings on a campus of a large organization. Petcharat et al. \cite{petcharat_assessment_2012} also found lighting energy savings on a group of buildings. Cabrera and Zareipour \cite{cabrera_data_2013} used data association rules to complete a similar study to find wasteful patterns. Geyer et al.  and Schlueter et al. test various clustering strategies to group various buildings within a Swiss alpine village according to their applicability for retrofit interventions \cite{geyer_application_2016} and thermal micro-grid feasibility \cite{schlueter_analysis_2016}.

\section{Operations, Optimization, and Controls}
\label{Operations}
Unsupervised techniques focused on individual buildings themselves are placed in the category for building operations, optimization, and control. This class contains the largest number of publications, and it incorporates a wider range of applications. It is differentiated from Section \ref{AnomalyDetection} in that the applications are not as focused on detecting and fixing the anomalous behavior. This section evaluates publications within the sub-categories of occupancy detection, retrofit analysis, controls, and energy management. Table \ref{fig:operations_table} outlines the publications in this section and their key attributes.


