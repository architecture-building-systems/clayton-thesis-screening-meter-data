\subsection{Discussion}
\label{sec:scalabilitydiscussion}

This section discusses the creation of additional information about smart meter by extracting characteristics from the high-frequency time-series measurements. Based on a classification test using almost 9,600 labeled smart meter accounts, the accuracy of predicting building type is improved (based on SIC 1-Digit category) by over 27\% over a conventional baseline. 

We then aggregated data about energy efficiency measures implementation and classified almost 1,600 accounts into Good, Average, and Poor performing classes according to pre and post-measure consumption. A classification model is developed that improves the ability to predict measure implementation class success by 18\% over a baseline. Additionally, there was only a 20\% error rate in differentiating between Good and Poor performing measures.

The biggest opportunity ahead is to apply to unlabeled smart meter accounts to help characterize missing meta-data and predict measure implementation success for future projects. Much work is also yet to be done to improve the models and input information to bring the overall prediction accuracies higher in absolute terms. Model prediction can also be improved incrementally as the AMI and measures implementation data are better integrated. 
%A promising area in prediction accuracy is in the use of better daily and weekly pattern recognition techniques.

% Recommender-style Implementation on the Unlabeled Data
% As stated in the summary, the best logical next step is to apply all the models to the 30,000 unlabeled smart meter to begin assigning the probabilities of meta-data and measure success. This could by providing account managers with a list of the top 5\% potential successful accounts based on measures and the top 3 most likely guesses as to what type of building it is based on SIC codes. This could be a type of ‘recommender’ system similar to what Amazon uses to suggest purchases based on previous history. 

% KITT and AMI Database Integration Efforts
% Combining the KITT and AMI is extremely valuable and is a task that VEIC is uniquely positioned to explore due to its record of measure implementation. The preliminary results show a non-trivial improvement in the ability to accurately predict whether a measure implementation will be successful. A two-fold improvement in labeled data can occur going forward: the first is that AMI and KITT data will continue accumulate, thus increasing the time range of useful data for pre and post measure implementation (currently it is 1 year – Oct 2013 – Oct. 2014), the second is further mapping of the KITT accounts to the AMI accounts.

% Refinement of the Temporal Feature Library to Improve Model Prediction Accuracy
% Another immediate obvious goal is to continue to refine and grow the temporal feature set. At the moment, many of the features are single value aggregations of the entire time range. Sub-sequence aggregation has the potential to further improve accuracy via assigning a value to each week, month or year within the time range. This approach creates a much larger and richer feature set. 

