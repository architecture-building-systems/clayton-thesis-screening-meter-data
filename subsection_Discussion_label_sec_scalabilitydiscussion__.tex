\subsection{Discussion}
\label{sec:scalabilitydiscussion}

This section discusses the creation of additional information about smart meter by extracting characteristics from the high-frequency time-series measurements. Based on a classification test using almost 9,600 labeled smart meter accounts, the accuracy of predicting building type is improved (based on SIC 1-Digit category) by over 27\% over a conventional baseline. 

We then aggregated data about energy efficiency measures implementation and classified almost 1,600 accounts into Good, Average, and Poor performing classes according to pre and post-measure consumption. A classification model is developed that improves the ability to predict measure implementation class success by 18\% over a baseline. Additionally, there was only a 20\% error rate in differentiating between Good and Poor performing measures.

The biggest opportunity ahead is to apply to unlabeled smart meter accounts to help characterize missing meta-data and predict measure implementation success for future projects. Much work is also yet to be done to improve the models and input information to bring the overall prediction accuracies higher in absolute terms. Model prediction can also be improved incrementally as the AMI and KITT Databases grow and better data integration is achieved. A promising area in prediction accuracy is in the use of better daily and weekly pattern recognition techniques.
