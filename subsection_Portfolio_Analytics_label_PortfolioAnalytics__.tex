\subsection{Portfolio Analytics}
\label{PortfolioAnalytics}

Portfolio analysis is a domain in which a large group of buildings, often located in the same geographical area or owned or managed by the same entity, are analyzed for the purpose of managing or optimizing the group as a whole. Each subsection covers the publications reviewed in this domain that fall into three categories: characterization, classification, and targeting.

\subsubsection{Portfolio Characterization}
Publications that address the characterization of a portfolio of buildings include unsupervised techniques meant to evaluate and visualize the range of behaviors and performance of the group. A majority of the techniques utilized are either clustering or visual analytics that provide a model of exploratory analysis that enable further steps. Seem \cite{seem_pattern_2005} produced an influential study that extracts days of the week with similar consumption profiles. Further clustering work was completed by An et al. \cite{an_estimation_2012} to estimate thermal parameters of a portfolio of buildings. Lam et al. \cite{lam_principal_2008} used Principal Component Analysis to extract information about a group of office buildings. Approaches focused on visual analytics and dashboards were completed by Agarwal et al. \cite{agarwal_energy_2009}, Lehrer \cite{lehrer_research_2009}, and Lehrer and Vasudev \cite{lehrer_visualizing_2011}. Granderson et al. \cite{granderson_building_2010} completed a case study-based evaluation of energy information systems, in which some methods combine some unsupervised approaches with visualization. Diong et al. \cite{diong_establishing_2015} completed a case study as well focused on a specific energy information system implementation. Mor\'an et al. \cite{moran_analysis_2013} and Georgescu and Mezic \cite{georgescu_site-level_2014} developed hybrid methods that employed visual continous maps and Koopman Operator methods respectively to visualize portfolio consumption. Miller et al. \cite{miller_forensically_2015,miller_automated_2015} completed two studies focused on the use of screening techniques to automatically extract diurnal patterns from performance data and use those patterns to characterize the consumption of a portfolio of buildings. Yarbrogh et al. used visual analytics techniques to analyze peak demand on a university campus \cite{yarbrough_visualizing_2015}.

\subsubsection{Portfolio Classification}
The concept of classifying buildings within a portfolio supplements the characterization techniques by assigning individual buildings to subgroups of relative performance for the purpose of benchmarking or decision-making. Santamouris et al. \cite{santamouris_using_2007} produced a report using clustering and classification to assign schools in Greece to subgroups of similar performance. Nikolaou et al. \cite{nikolaou_application_2012} and Pieri et al. \cite{pieri_identifying_2015} further extended this type of work to office buildings and hotels. Heidarinejad et al. \cite{heidarinejad_cluster_2014} released an analysis of clustered simulation data to classify LEED-certified office buildings. Ploennigs et al. \cite{ploennigs_e2-diagnoser:_2014} created a platform for monitoring, diagnosing and classifying buildings and operational behavior within a portfolio to quickly visualizing the outputs.

\subsubsection{Portfolio Targeting}
Targeting is a concept that builds upon characterization and classification to identify specific buildings or measures to be implemented in a portfolio to improve performance. These publications are differentiated in that specific measures are identified in the analysis. Sedano et al. \cite{sedano_improving_2009} uses Cooperative Maximum-Likelihood amongst other techniques to evaluate the thermal insulation performance of buildings. Gaitani et al. \cite{gaitani_using_2010} used PCA and clustering to target heating efficiency in school buildings. Bellala et al. \cite{bellala_towards_2011} used various methods to find lighting energy savings on a campus of a large organization. Petcharat et al. \cite{petcharat_assessment_2012} also found lighting energy savings on a group of buildings. Cabrera and Zareipour \cite{cabrera_data_2013} used data association rules to complete a similar study to find wasteful patterns. Geyer et al.  and Schlueter et al. test various clustering strategies to group various buildings within a Swiss alpine village according to their applicability for retrofit interventions \cite{geyer_application_2016} and thermal micro-grid feasibility \cite{schlueter_analysis_2016}.