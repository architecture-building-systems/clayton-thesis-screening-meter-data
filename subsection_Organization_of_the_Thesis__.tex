\subsection{Organization of the Thesis}
\label{sec:organization}

The remainder of this thesis is organized as follows. The research context of contemporary statistical learning and visual analytics techniques as applied to building performance is reviewed in Section \ref{sec:litreview}. This section has a special focus on unsupervised learning techniques as they are a strong basis for many of the temporal features extracted. Section \ref{sec:casestudies} provides an overview of the process of collecting data and insight from a series of case studies from around the world. Data from over 1200 buildings was collected on-site or through various open web portals and 507 were selected for further analysis. Section \ref{sec:featureextraction} provides an in-depth overview of each of the temporal mining techniques implemented on the case study buildings, including explanatory visualizations of the range of values across the tested time range. Section \ref{sec:characterization} discusses the use of these features for the characterization of objectives such as predicting building use type, performance class, and operations type. Section \ref{sec:scalability} focuses on the use of a subset of temporal features in the industry classification and prediction of energy savings measures of close to 10,000 buildings with AMI data available. Finally, Section \ref{sec:conclusion} provides concluding remarks to understand the overall results of the thesis and future directions to pursue using the outlined techniques.
