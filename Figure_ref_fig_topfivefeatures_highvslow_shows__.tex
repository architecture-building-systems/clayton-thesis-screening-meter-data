Figure \ref{fig:topfivefeatures_highvslow} shows the probability distributions of the top five performing features in predicting high versus low consumption. The number one top feature for differentiating between these classes is the daily in-class similarity feature called \emph{BG_jmotif_inclasssim_24_6_6} that is generated by the \emph{jmotif} process. This feature informs us that buildings from all classes that have the highest average daily pattern similarity to their peers are often also amongst the highest consuming buildings in their class. Buildings that are on average less similar in their daily patterns to their class are often in a lower percentile of consumption. This fact suggests that many buildings that are misclassified are lower consumers of electricity. The second and fourth features are daily statistical ratios. Buildings with higher consumption tend to have more \emph{flat} profiles, likely due to a higher base load during unoccupied periods. The third top classifier is also created using the \emph{jmotif} library and it suggests that a building that whose minimum daily pattern specificity across the year is an indicator of higher than average consumption. 