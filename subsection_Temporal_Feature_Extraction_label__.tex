\subsection{Temporal Feature Extraction}
\label{temporalfeatureextraction}
Modern, whole building electrical meters measure and report raw, sub-hourly, time-stamped data. Large amounts of intrinsic information can be extracted from temporal data to characterize a commercial building. The harvest of these information can assist in the implementation of conventional analysis techniques, as inputs to classify or benchmark a building, or to predict whether a building is a good candidate for certain energy savings measures. In order to extract information solely from these sensors, new features can be created from these raw data. These features are designated as temporal as they summarize behavior occurring in time-series data. To illustrate the concept of temporal features qualitatively, Figure \ref{fig:electricalmeters_oneyear} shows four example hourly electrical meters from different buildings. Even to the untrained eye, these data streams show obvious differences in the way each building operates. Building A seems to be an extremely consistent consumer of energy across the entire year. There are no steady-state shifts in operation and seemingly no influence from outside factors. Qualitatively, this data stream can be thought of as \emph{consistent} or \emph{predictable}. Building B is similar in operation, but has an obvious influence from an external factor in the summer months. It is safely assumed that the consumption of this building is weather-dependent and it has some kind of cooling system. Building C illustrates behavior that has \emph{shifts} in consumption over the course of the year. This observation implies that this building has different schedules over the course of a year. Building D seems to have combinations of all of these attributes, with no obvious phenomenon that stands out. 
