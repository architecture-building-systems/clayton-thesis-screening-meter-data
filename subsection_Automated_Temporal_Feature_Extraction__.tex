\subsection{Automated Temporal Feature Extraction}
Many modern buildings have a whole building electrical meter that is automatically storing raw, sub-hourly, time-stamped data. In order to extract information solely from these sensors, new features must be created from these raw data.  These features are designated as temporal as they summarize behavior occurring in the high-frequency, time-series data. Figure \ref{fig:temporalfeatures} illustrates the categories of temporal features created in this effort.

One emerging trend is that "data mining algorithms should have as few parameters as possible, ideally none. A parameter-free algorithm prevents us from imposing our prejudices and presumptions on the problem at hand and let the data itself speak to us \cite{Keogh:2004vp}." This approach is known as \emph{parameter-free} or \emph{parameter-light} data mining. The efficacy of these algorithms has been proven comparable or better than many more complex, traditional time-series data mining approaches \cite{Keogh:2004vp}. The primary goal of the features outlined is robustness despite a scarcity of meta data.