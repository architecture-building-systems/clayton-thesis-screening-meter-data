\subsection{Overview of Temporal Features Influence on Objectives}
\label{sec:overviewuroftemporalfeat}

% The development of the features in this study was completed with the intent of capturing much of the common building performance characterization techniques commonly found in the research literature. Additionally, several novel pattern-based techniques were implemented and tested. It is of interest to test this features set against a benchmark set of temporal data mining techniques to understand how thorough the proposed set of metrics are at characterizing the data. This benchmark comparison is done using the \emph{highly comparative time-series analysis} feature library \cite{Fulcher_2013}. This library includes over 9000 temporal features covering a vast range of algorithm types.

% %data for the following table is from: http://rsif.royalsocietypublishing.org/content/royinterface/suppl/2013/04/03/rsif.2013.0048.DC1/rsif20130048supp2.pdf

% \begin{table} 
%     \begin{tabular}{ c c c }
%         \textbf{Feature Category} & \textbf{Brief Description} & \textbf{Number of Features} \\
%         Test & Test & Test \\ 
%         Test &  &  \\ 
%          &  &  \\ 
%          &  &  \\ 
%          &  &  \\ 
%          &  &  \\ 
%          &  &  \\ 
%          &  &  \\ 
%          &  &  \\ 
%          &  &  \\ 
%          &  &  \\ 
%     \end{tabular} 
% \end{table}

