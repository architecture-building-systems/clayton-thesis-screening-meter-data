\chapter{Kurzfassung}

Diese Studie behandelt das Sichten, Sortieren und Bearbeiten charakteristischer Zeitreihen aus stark wachsenden Quellen fu\"r rohe Sensordaten in kommerziellen Gebäuden. Ein zweistufiges Vorgehen wird präsentiert, das statistische, modellbasierte und Muster-basierte Verhaltensweisen von zwei Datensätzen extrahiert. Der erste Datensatz beinhaltet Daten von 507 kommerziellen Gebäuden, zusammengetragen aus verschiedenen Fallbeispielen und online Datenquellen aus der ganzen Welt. Der zweite Datensatz beinhaltet Daten von Advanced Metering Infrastructure (AMI) von 1,600 Gebäuden. Das Ziel der vorgestellten Methode ist das Reduzieren benötigter Experteneingriffe, um gemessene Rohdaten benutzen zu können zum Erhalten von Information wie Gebäudenutzungstyp, Performance Klasse und Betriebsverhalten. Im ersten Schritt, dem Extrahieren von Charakteristiken, werden durch das Benutzen einer Bibliothek von Data Mining Techniken verschiedene Phänomene aus den Rohdaten herausgefiltert. Dieser Schritt transformiert quantitative Rohdaten zu qualitativen Kategorien, die durch Heat Map Visualisierungen präsentiert und interpretiert werden. Im zweiten Schritt, der Datenuntersuchung, wird eine Supervised Learning Technique auf die Möglichkeit hin getestet, den wichtigsten Charakteristiken aus dem ersten Schritt eine Auswertung der Auswirkungen zuzuordnen. Um das Hochskalieren für heterogene Gebäudeparks zu untersuchen wird die Wirksamkeit getestet, variable Kausalzusammenhänge der charakterisierten Performance zu schätzen. In den Fallstudien im ersten Datensatz war die Bestimmung des primären Gebäudenutzungstyps dreimal treffender, die Bestimmung der Performance Klasse fast zweimal treffender und die Bestimmung des Betriebsverhaltenstyps über viermal treffender als für ein Basisvorgehen. Für die AMI Daten wurde die Charakterisierung der Standard Industrie Klasse  um 27\% verbessert, die Prognose der Erfolgsrate von Energiesparmassnahmen um 18\% verbessert. Interviews mit Akteuren von mehreren Schulanlagen werden diskutiert bezüglich ihrer qualitativen Einblicke und bezüglich der Nützlichkeit der vorgestellten Ansätze im Kontext des Betriebs von Schulanlagen. 
