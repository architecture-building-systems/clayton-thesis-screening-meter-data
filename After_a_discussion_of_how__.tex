After a discussion of how different use types of buildings are characterized using temporal features, the concept of mis-classified buildings was introduced. Misclassification of buildings pertains to when the primary use type of the building doesn't match the temporal features of the electrical consumption, particularly the daily and weekly patterns of use. Figure \ref{fig:specificity_high} was designed to illustrate this concept. This figure contains a subset of the case study buildings within the office, university classroom, and university laboratory categories. The pattern specificity for offices, classrooms and laboratories was calculated for each building as shown in the first three columns of the graphic. They are clustered according to their similarity with red indicating low values and blue indicating high values. The column on the far right indicates the use type classification for each building. The laboratories are yellow, classrooms are blue, and offices are green. It can be seen that there are distinct clusters of building types and a few regions in which there