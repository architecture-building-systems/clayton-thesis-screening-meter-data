\section{Introduction}
\label{sec:intro}

The creation and consolidation of measured sensor sources from the built environment and its occupants is occurring on an unprecedented scale. The Green Button Ecosystem now enables the easy extraction of performance data from over 60 million buildings\footnote{According to: http://www.greenbuttondata.org/}. Advanced metering infrastructure (AMI), or smart meters, have been installed on over 58.5 million buildings in the US alone\footnote{As of 2014, according to: http://www.eia.gov/tools/faqs/faq.cfm?id=108&t=3}. A recent press release from the White House summarizes the impact of utilities and cities in unlocking these data \cite{the_white_house_fact_2016}. It announces that 18 utilities, serving more than 2.6 million customers, will provide detailed energy data by 2017. This study also suggests that such accessibility will enable improvement of energy performance in buildings by 20\% by 2020. A vast majority of these raw data being generated are sub-hourly temporal data from meters and sensors.

In order to understand the exponential magnitude of this data source growth in the building industry, one can estimate the amount of measurements being generated by these sensors. The United States context has public data available to create a set of assumptions to roughly quantify this growth. Before the widespread use of digital building automation systems, buildings were controlled either manually or using pneumatic controls and building electrical use was measured and reported monthly. According to the Commerial Building Energy Consumption survey, there were over 4.5 million commercial buildings in the United States in 1996. The theoretical amount of data from monthly electrical meters for all of these buildings for one year would be 54 million measurements. In about 2007, electrical meters with the capability to capture and store data at 15 minute frequencies were introduced into the market and 7 million were installed on all building types \footnote{http://www.edisonfoundation.net/iei/Documents/IEI_SmartMeterUpdate_0914.pdf}. If ones assumes that the proportion of these meters that are commercial is similar to today\footnote{About 11.2\% according to: http://www.eia.gov/tools/faqs/faq.cfm?id=108&t=3}, that would result in approximately 784,000 buildings creating 27.4 billion measurements per year. By 2014, AMI meters have been installed on 6.53 million commercial buildings resulting in 228 billion measurements per year. The exponential magnitudes of growth of these data can be seen in Figure \ref{fig:datagrowth}. This discussion ignores the concept of accessibility which has also vastly improved due to the technology.
%  \cite{energy_information_administration_how_2015} \cite{james_manyika_unlocking_2015}
%The Internet-of-Things (IoT) movement provides an array of low-cost sensors, data acquisition devices, and cloud storage. A recent study has predicted a \$70-150 billion impact of IoT in offices and \$200-350 billion in homes.