\subsection{Statistics-based Features}
\label{sec:statisticsfeatures}

In order to get a sense of what a temporal feature is in the context of electrical meters for buildings, the first extracted metric is one of the most commonly calculated for building performance analysis: the consumption magnitude of electricity normalized by the floor area of the building. This metric seeks to provide a basis of comparison between buildings and is used as a key metric within numerous benchmarking and performance analysis techniques. Figure \ref{fig:normalizedmag} illustrates a single building example of this metric per hour across a time range of two weeks at the end of the year. The top line chart of this figure shows the magnitude of hourly electrical consumption for one of the case study buildings. The middle portion of the figure repeats this information in the form of a color-based, one dimensional heatmap. In this example, the daily weekday profiles manifest themselves as light-colored bands and weekend and unoccupied periods as darker bands. The color bar at the bottom of the figure is key in interpreting the color values. This figure is an example of a single building demonstration of this particular feature and is a type of graphic that is used throughout this entire section.