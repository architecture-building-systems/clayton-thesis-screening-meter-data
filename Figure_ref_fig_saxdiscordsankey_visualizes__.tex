Figure~\ref{fig:saxdiscordsankey} visualizes the frequency of the SAX strings and substrings in the form of an augmented suffix tree. Suffix trees have been an integral part of string manipulation and mining for decades \citep{weiner_linear_1973}. Augmented suffix trees enable a means of visualizing the substring patterns to show frequency at each level. This figure incorporates the use of a sankey diagram to visualize the tree with each substring bar height representing the number of substring patterns existing through each window of the day-types. The more frequent patterns are categorized as \emph{motifs}, or patterns which best describe the average behavior of the system. One can see the patterns with the lower frequencies and their indication as \emph{discords}, or subsequences that are least common in the stream.  

Heuristically, we set a decision threshold to distinguish between motifs and discords. This threshold can be based on the word frequency count for each pattern as a percentage of the count of all observations. This threshold can be tuned to result in a manageable number of discord candidates to be further analyzed. More details pertaining to setting this threshold will be discussed the applied case studies.

% The extracted SAX words are grouped according to frequency and a threshold is chosen to differentiate the profiles to consider as motif candidates or as low frequency day types to tag for further investigation.

In the two week example, this process yields two patterns which have a frequency greater than one and thus are the motif candidates. A manual review of the data confirms that those patterns match with a normally expected profile for a typical weekday ($acca$) and weekend ($aaaa$). The less frequent patterns are tagged as discords and can be analyzed in more detail. In this case it can be determined that the patterns $abba$, $abca$, and $acba$, despite being infrequent, are not abnormal due to the occupancy schedule for those particular days. Pattern $ccba$, however, is not explainable within the scheduling and is due to a fault causing excessive consumption in the early morning hours.

This step leads into the next phase of the process focused on further aggregating the motif candidates of the dataset. The size and number of potential motif filtered in this step will give an indication of the number of clusters that will likely pick up meaningful structure from the dataset.