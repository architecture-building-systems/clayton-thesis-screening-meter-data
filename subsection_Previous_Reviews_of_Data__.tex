\subsection{Previous Reviews of Data Analytics in Buildings}
\label{sec:previousreviews}

Various reviews have been completed that overlap with this section. Most of them are designed to focus on a single core domain of research; the main two areas are building operations analysis and smart grid optimization. One of the earliest reviews of artificial intelligence techniques for buildings was completed in 2003 by Krarti and covered both supervised and unsupervised methods \cite{krarti_overview_2003}. Dounis updated this work and focused on outlining specific techniques in detail \cite{dounis_artificial_2010}.  Reddy's seminal book about a large variety of analysis techniques for energy engineers includes chapters on clustering and unsupervised methods specifically \cite{reddy_applied_2011}. Lee et al. describe a variety of retrofit analysis toolkits which incorporate unsupervised and visual analytics approaches in a practical sense \cite{lee_energy_2015}. Ioannidis et al. created a large ontology of data mining and visual analytics for building performance analysis, however with a strong focus on the techniques and not examples of works using them \cite{ioannidis_big_2015}. From the utility and power grid side, Morias et al. created a general overview of various data mining techniques as focused on power distribution systems \cite{morais_overview_2009}. Chicco covered clustering methods specifically focused on load profiling tasks \cite{chicco_overview_2012}. Zhou et al. included the concept of customer load classification  \cite{zhou_review_2013}.\\