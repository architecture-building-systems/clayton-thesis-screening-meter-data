\emph{DayFilter} is applied on a large energy performance datasets to demonstrate the usability and results in real-life scenarios. The process is applied to a 70,000 square meter international school campus in the humid, tropical climate of Singapore. It was built in 2010 and includes a building management system (BMS) with over 4,000 measured data points taken at 5 minute intervals from the years of 2011-2013 - resulting in close to 800 million records of raw data. This collection includes 120 power meters and 100 water meters in the energy and water management system. The data from this study are a seed dataset in an open repository of detailed commercial building datasets \cite{miller_seed_2014}. The primary energy load for this campus is the chilled water plant which consumes almost \EUR{40,000}/month in electricity.

The chilled water plant electricity consumption is targeted in this case due to its importance in this climate and the potential savings opportunities available through chilled water plant optimization. Measured kilowatt-hour (kWh) and kilowatt (kW) readings were taken from July 12, 2012 to October 29, 2013 with 474 total daily profiles analyzed. Figure~\ref{fig:sankeyheatmap1} illustrates a sankey diagram with heatmap of the output of the \emph{DayFilter} process with parameters set to $A$=3 and $W$=4. The discord and motif candidates are separated in this case according to a decision threshold which quantifies a discord as a day-type with a frequency count less than 2\% of total days available. This distinction results in 39 days with patterns tagged as discord candidates, which is 8.2\% of the total days in the dataset. 