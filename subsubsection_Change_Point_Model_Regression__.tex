\subsubsection{Change Point Model Regression}
\label{sec:changepointmodels}

Another means of performance modeling that takes weather characterization into consideration is the use of linear change point models. The outputs of these models can be interpretable in approximating the amount of energy being used for heating, ventilation, and air-conditioning (HvAC). This type of model has its basis in the previously-mentioned PRISM method and has been continuously utilized, recently by Kissock and Eger \cite{Kelly_Kissock_2008}. This multi-variate, piece-wise regression model is developed using daily consumption and outdoor air dry-bulb temperature information. A linear regression model is fitted to data detected to be correlated with outdoor dry-bulb air temperature, either positively for cooling energy consumption or negatively for heating energy consumption. For example, as the outdoor air temperature climbs above a certain point, the relationship between electrical consumption and every degree increase in temperature should be a linear line with a certain slope if the building has an electrically-driven cooling system. The point at which this change occurs is considered the cooling balance point of the building and the slope of the line is the rate of cooling energy increase due to outdoor air conditions. This example can be seen in Figure \ref{fig:changepointmodels}a in which the base load of the building is designated as $\beta_1$