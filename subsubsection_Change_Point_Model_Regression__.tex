\subsubsection{Change Point Model Regression}
\label{sec:changepointmodels}

Another means of performance modeling that takes weather characterization into consideration is the use of linear change point models. The outputs of these models can be interpretable in approximating the amount of energy being used for heating, ventilation, and air-conditioning (HvAC). This type of model has its basis in the previously-mentioned PRISM method and has been continuously utilized, recently by Kissock and Eger \cite{Kelly_Kissock_2008}. This multi-variate, piece-wise regression model is developed using daily consumption and outdoor air dry-bulb temperature information. A linear regression model is fitted to data detected to be correlated with outdoor dry-bulb air temperature, either positively for cooling energy consumption or negatively for heating energy consumption.