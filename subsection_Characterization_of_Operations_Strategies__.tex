\subsection{Characterization of Operations Strategies}
\label{sec:operations_strategies}

The final characterization objective for the case studies is the ability for the temporal features to classify buildings from the same campus, and thus buildings that are being operated in similar ways. This characterization takes to into account the similarity in occupancy schedules, patterns of use, and other factors related to how a building operates. Like the performance classes, this type of classification is more important in understanding the features that contribute to the differentiation, rather than the classification itself. Seven campuses were selected from the 507 buildings to create seven \emph{groups} of buildings to characterize the difference between their operating behavior. Features were removed for this objective that are indicators of weather sensitivity as these would be related to the location of the buildings, and thus, the campus that they're located on. Figure \ref{fig:operations_classification} illustrates the results from the random forest model trained on these data. The accuracy of this model is 80.5\% as compared to a baseline of 16.9\%. The model is very good at predicting some of the groups, such as groups 1-4, which more deficient in others, such as 5-7. 