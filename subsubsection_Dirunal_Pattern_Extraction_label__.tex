\subsubsection{Dirunal Pattern Extraction}
\label{sec:dayfilter}

Towards the development of diurnal motif and discord extraction, a new techniques was developed as an application of temporal data mining to building performance data. It is a process called \emph{DayFilter} and it includes five steps designed to incrementally filter structure from daily raw measured performance data. These steps, as seen in Figure~\ref{fig:process}, are intended to bridge the gap between contemporary top-down and bottom-up techniques. The arrows in the diagram denote the execution sequence of the steps. Note that steps 3, 4, and 5 produce results applicable to the implementation of bottom-up techniques.  

Whole building and subsystem metrics are targeted for analysis in order to determine high level insight. The process begins with a data preprocessing step which removes obvious point-based outliers and accommodates for gaps in a univariate dataset of variable length. Next, the raw data is transformed into the SAX time-series representation for dimensionality reduction by creating groups of SAX words from daily windows. This step enables the quick detection of \emph{discords}, or daily patterns of performance that fall outside what is considered normal in the dataset according to the frequency of patterns. The discords are filtered out for future investigation while the remaining set of SAX words is clustered to create performance \emph{motifs}, or the most common daily profiles. The additional clustering step beyond the SAX transformation and filtering adds the ability to further aggregate daily profiles beyond the SAX motif candidates. These clusters are useful in characterizing what can be considered \emph{standard} performance. Finally, these data are presented using visualization techniques as an aid to identify and interpret the questionable discords and the common clusters. In the following simplified example, we detail each of these steps. The input parameter selections in this section are based on suggestions from other studies using SAX aggregation and clustering approaches. Additional discussion of parameter selection is presented in Section \ref{sec:parameters}.