\subsection{Conventional Commercial Building Characterization}
A comprehensive study of building performance analysis was completed by the California Commissioning Collaborative (CACx) to characterize the technology, market, and research landscape in the United States. Three of the key tasks in this project focused on establishing the state of the art \cite{effinger_building_2010}, characterizing available tools and their the barriers to adoption \cite{ulickey_building_2010}, and establishing standard performance metrics \cite{greensfelder_building_2010}. These reports were accomplished through investigation of the available tools and technologies on the market as well as discussions and surveys with building operators and engineers. The common theme amongst the interviews and case studies was the \emph{lack of time and expertise} on the part of the involved operations professionals. The findings showed that installation time and cost was driven by the need for an engineer to develop a full understanding of the building and systems. These results are interpreted as a latent need for techniques that take into consideration the people, process, and philosophy aspects of the performance analysis equation \cite{miller_applicability_2013}. 

Figure \ref{fig:convfeatures} illustrates a hierarchical node diagram of the features, or meta data, about a building that is often necessary to accumulate in order to perform conventional analysis from the literature. Much of these data are needed when creating an energy simulation model, when setting thresholds for automated fault detection and diagnostics, or benchmarking a building. When performing analysis on a single building, these meta-data might be easy to accumulate. However, when such a process is scaled across hundreds, or potentially thousands of buildings, collection of these data is not a trivial procedure. 
%This reality is reinforced through site interviews with campus-level energy managers as described later in this thesis.

%force chart data and viz in /Dropbox/04_ANALYSIS