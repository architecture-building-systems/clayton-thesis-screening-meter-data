In this study, the goal is not to use SAX-VSM to classify each data stream, but to instead extract temporal features that can be used to characterize each stream. Thus, the in-class cosine similarity is calculated for each building's data set as compared to the class it was assigned. This process is not conventional from the classification sense as it is considered over-fitting due to all samples being included in the training set. This situation is tolerated in this analysis as we simply want to quantify how much the patterns of use for a building compare to those of its labeled \emph{peers}.