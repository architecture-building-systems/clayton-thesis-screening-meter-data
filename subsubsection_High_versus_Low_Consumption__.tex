\subsubsection{High versus Low Consumption Comparison}
\label{sec:highvslow}

The two classifications chosen for this objective are intuitively the \emph{high} and \emph{low} consuming buildings. This part of the analysis gives a more in-depth perspective of exactly which features are most important in the differentiation between these two types of buildings. This understanding provides insight on potentially what behavior in a building results in good or poor performing buildings. Once again, the highly comparative time-series analysis (hctsa) code repository is used for this process. Figure \ref{fig:featurecluserting_performanceclass} is a correlation matrix showing the top forty features as determined by hctsa according to the in-sample linear classification performance. Eight clusters of features are detected with respect to discriminating between high and low consumption. The first set of correlated features seen in the upper left corner of the figure contain a mix of statistical and daily pattern-based features. The second cluster contains a set of four features related to daily ratios. The third and largest cluster is mostly statistical and daily ratio-based features. The fourth, sixth, seventh, and eighth clusters all contain mostly in-class similarity and temporal features created using \emph{jmotif}. These features are an indicator of how well a building's patterns fit within its own class. An interesting aspect of these features is their lack of correlation with the rest of the larger set. This situation indicates that they are capturing unique behavior, not picked up by others in the set. These clusters are also relatively small with only one to four members. The sixth cluster contains a set of features that are mostly generated by the \emph{stl} decomposition models.

