\subsubsection{Discussion with Campus Case Study Subjects}
\label{sec:buildinguseclassificationdiscussion}

Previously, an example of how to characterize building use type was illustrated using a random forest model and various feature importance techniques. In this subsection, a discussion is presented of how this type of characterization can be useful in a practical setting. In the case study interviews, the topic of benchmarking of buildings was discussed. One of the issues presented to the operations teams was the concept of not having a complete understanding of the way the buildings on their campus were being used. For example, several of the campuses have a spreadsheet outlines various meta data about the facilities on campus. This spreadsheet, in many cases, includes the \emph{primary use type} of the building. It was found that this primary use type designation is often loosely based on information from when the building was constructed or through informal site survey. In other situations, the building has an accurate breakdown of all the sub-spaces in the building and approximately what the spaces are being used for. In these discussions, the idea was presented that building use type characterization could be used to automatically determine whether the labels within these spreadsheets actually aligned with the patterns of use characterization using the temporal feature extraction. This proposal was met some positive feedback, albeit there was a hesitation to fully confirm that this process would be entirely necessary if labor was directed to do the same task.\\
\\
Many of the case study subjects then where shown a series of graphics designed to tell the story of building use type characterization in an automated way. Figure \ref{fig:buildingusebreakdownforcasestudies} is the first figure shown to the subjects. This figure illustrates several of the most easily understood temporal features and how they break down across the various building use types. This graphic was created using the data for a particular case study, therefore more separation between the classes exist than in the prediction of classes found in the previous section. Discussions using this graphic first centered around the first feature: \emph{Daily Magnitude per Area}. It was intuitive to most participants that a university laboratory has more  and primary/secondary schools have less consumption per area than the other use types. It is more surprising, however, that certain building use types are characterized well by other features, such as number of breakouts with primary/secondary schools and daily and weekly specificity with university dormitories.

% Grouping of characteristically similar buildings visualize and explore the \emph{phenotypes} of buildings. The objective of this section is focused on the identification of the main modes in which in building use-type manifests itself and the identification of \emph{misfit} buildings, or those that don't seem to act like they're supposed to.