\section{Characterization of Energy-Savings Measure Implementation Success}
\label{sec:scalability}

In the previous sections, the process of temporal feature extraction and interpretation is implemented on a set of 600 buildings. One could argue that conventional analysis and meta-data collection for a set of buildings at this level is reasonable. It is much harder to make this statement when discussing the millions of buildings with smart meter data. In this section, execution of the temporal feature extraction process is applied to a data set of close to 10,000 buildings that have been aggregated by a third-party organization on behalf of local utilities. The utilization goal of these data is to supplement a process of targeting buildings for energy savings implementation measures. Utilization of temporal features is discussed in the context of assisting to label the approximate building use type according to building use-type and predicting measure success implementation through a combination of smart meter data and past project experience.

% Out of the 40,000 buildings, only around 9,600 have a significant amount of known meta-data. This situation leaves over 30,400 accounts as essentially unlabeled and thus, very difficult to use in analysis.

% The other challenge is that VEIC has a rich repository of data from thousands of past implementation projects in a system known as KITT. There is a huge potential in using information from past projects to predict the success of future implementations. There are 3,000 AMI accounts with data overlapping with the KITT database between January 2013 and June 2015. Manual analysis of this data overlap is time consuming to predict measure success is not trivial.
