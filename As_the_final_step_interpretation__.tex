As the final step, interpretation and visualization are important for \emph{DayFilter} in order for a human analyst to visually extract knowledge from the results, and to make decisions regarding further analysis. We utilize insight from the \emph{Overview, zoom and filter, details-on-demand} approach \cite{shneiderman_eyes_1996} and the previously mentioned VizTree tool \cite{lin_visually_2004}. The hidden structures of building performance data is revealed through the SAX process and we use visualization to communicate this structure to an analyst. The process uses a modified sankey diagram to visualize the augmented suffix tree in a way which the count frequency of each SAX word can be distinguished.  Figure~\ref{fig:saxdiscordsankeyheatmap} shows how this visualization is combined with a heatmap of the daily profiles associated with each of the SAX words using the same two-week example data from Figures \ref{fig:saxcreation} and \ref{fig:saxdiscordsankey}. The sankey diagram is rearranged according to the frequency threshold set to distinguish between the motif and discord candidates.

In Figure~\ref{fig:saxdiscordsankeyheatmap}, the discords are shown as the top four days, Jan. 2, 3, 12, and 13 and the remaining days shown as more frequent potential motifs below. Each daily profile is shown adjacent to the right of the sankey diagram and is expressed as a color-based heatmap. Each horizontal bar of the heatmap is an individual day and they are grouped according to pattern with the associated legend informing the viewer the magnitude of energy consumption across the day. This visualization is designed to quickly present the patterns grouped according to a sort of hierarchy provided by the suffix tree. One can more easily distinguish seemingly \emph{normal} versus \emph{abnormal} behavior with this combination of visualizations. 