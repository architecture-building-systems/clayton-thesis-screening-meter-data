\section{Characterization of Building Use, Performance, and Operations}
\label{sec:characterization}

Visualization of temporal features on their own is a means of understanding the range of values of various phenomenon across a time range. This situation gives an analyst the basis to begin understanding what discriminates a building based on various objectives. The next step is to utilize the features to predict whether a building falls within a certain category and test the importance of various features in making that prediction. Understanding which features are most characteristic to a particular objective is the the key tenet of this study. In this section, three classification objectives are tested: 

\begin{enumerate}
\item Principle Building Use - The primary use of the building is designated by the principal activity conducted by percentage of space designated to that activity. It is rare for a building to be devoted specifically to a single task, and mixed-use buildings pose a specific challenge to prediction.
\item Performance Class - Each building is assigned to a particular performance class according to whether its area-normalized consumptiis in the the bottom, middle, or top 33\% percentiles within its principle building use type class.
\item General Operation Strategy - Buildings that are controlled by the same entity, such as those on a University campus, often have similar schedules, operating parameters, and use patterns. This objective tests to understand how well-defined these differences are between different campuses.
\end{enumerate}

For each objective, several steps are taken to predict each objective and then to investigate the influence of the input features on class differentiation:
\begin{enumerate}
\item A random forest classification model is built using subsets of the generated features to predict the objectives class
\item The classification model provides an indication of the ability of the temporal features in describing the class based on its accuracy
\item Input feature importance is calculated by the classification model for insight on what the most informative features are in predicting class
\item An in-depth analysis comparison of two of the classes within each objective is completed to further explore the attributes that characterize a building
\end{enumerate}

Random forest classification models were chosen based on their ability to model diverse and large data sets in a robust way \cite{Breiman}. These models use an ensemble of decision trees to predict various characteristic labels about each building based on the its features. The literature describes decision trees as the "closest to meeting the requirements for serving as an off-the-shelf procedure for data mining" \cite{hastie_elements_2009}. Figure 
