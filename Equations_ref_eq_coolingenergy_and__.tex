Equations \ref{eq:coolingenergy} and \ref{eq:heatingenergy} are used to predict energy energy consumption based on an outdoor air temperature, $T$. This equation can also predict the heating ($\beta_2(T - \beta_3)$) or cooling ($\beta_2(\beta_3 - T)$) components of the electrical consumption to a certain level of accuracy. 

\begin{equation}
\label{eq:coolingenergy}
E_c = \beta_1 + \beta_2(T - \beta_3)
\end{equation}

\begin{equation}
\label{eq:heatingenergy}
E_h = \beta_1 + \beta_2(\beta_3 - T)
\end{equation}

Figure \ref{fig:changepoint} illustrates a change point model fit on an office building in a continental climate that includes both heating and cooling seasons. It should be noted that the model is not perfectly characterizing the data due to two modes of daily operation; this situation is due to there being an offset between occupied and unoccupied operation. This model is used to generate features of approximate heating and cooling energy and in general the slopes of these two modes can safely be assumed to be similar in most cases. The Open Meter Python library is used to regress these models for each building in this study \footnote{http://www.openeemeter.org/}.