\section{Pattern-based Features}
\label{sec:weatherfeatures}

Temporal data mining for performance monitoring focuses generally on the extraction of patterns and model building of time series data. These techniques are, in some ways, similar to many existing building performance analysis approaches, however different concepts and terminology are used. Two key concepts to understand when applying data mining to buildings are that of \emph{motifs} and \emph{discords}. A motif is a common subsequence pattern that has the highest number of non-trivial matches \cite{Patel:2002bb}, thus, a pattern that is found frequently in the dataset. A discord, on the other hand, is defined as a subsequence of a time series that has the largest distance to its nearest non-self match \cite{Keogh:2005wd}. It is a subsequence of a univariate data stream that is least like all other non-overlapping subsequences and is, therefore, a rare pattern that diverges from the rest of the dataset. These definitions are more general than that of a \emph{fault} and therefore more appropriate for our goal of higher level information extraction with less parameter setting. In short, we want to efficiently find \emph{interesting or infrequent} behavior and not create a detailed list of specific problems that could be occurring in individual systems.

In order to work with common temporal mining approaches, Symbolic Aggregate approXimation (SAX) representation of time-series data \cite{Lin:2003wz} is used. SAX allows discretization of time series data which facilitates the use of various motif and discord detection algorithms. The process breaks time series data into subsequences which are converted into an alphabetic symbol. These symbols are combined to form strings to represent the original time series enabling various mining and visualization techniques. In terms of application, an example of a process using SAX-based techniques is the VizTree tool that uses augmented suffix tree visualizations designed for usability by an analyst \cite{Lin:2004wv,Lin:2005bi}. A specific application of VizTree is the analysis of collected sensor data from an impending space craft launch in which thousands of telemetry sensors are feeding data back to a command center where experts are required to interpret the data. Visualization and filtering tools are needed that allow a natural and intuitive transfer of mined knowledge to the monitoring task. Human perception of visualizations and the algorithms behind them must work in unison to achieve understanding of large amounts of novel data streams. 

% SAX has been used on building performance data before in a few studies focused on data center chilled water plants and it was found effective in detecting the most efficient control strategies \cite{Patnaik:2009uk}. The same research was used to create a visual exploration tool of high frequency time series data \cite{Hao:2012go}. Despite these efforts, our review of the literature found a lack of tools or processes similar to the VizTree tool for day-types that fit in our targeted context of bridging the performance gap. We will introduce a new process focused on combining temporal approximation, filtering, and visualization.