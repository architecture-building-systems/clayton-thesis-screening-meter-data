\subsubsection{Case Study #1} 
\label{sec:casestudy1}
The first case study is a campus in a continental climate in the Midwest region of the United States. It is a university with 226 buildings spread across two main campuses. All together, these buildings have a total floor area over 2.3 million square meters (25 million square feet). An initial interview was conducted with the lead statistician of the facilities management in March 2015. Information was gathered on the building and energy management systems of the campus and a discussion regarding the typical utilization of the data was conducted. It was found that there are over 480 electrical meters on the campus and that these data were primarily used for billing of the individual academic departments. They have a custom metering data management platform with some capabilities for data export. A second site visit was conducted in June 2015 to facilitate collection of a sample one year data set. In this site visit, a facilities management professional with experience in SQL databases was able to directly query the underlying back-end of the energy management system in order to extract one year of raw data from all of the metering infrastructure on the campus. An accompanying meta-data spreadsheet was discovered that included information on floor area, primary space usage, EnergyStar score, and address. These data were then used for the analysis and feature extraction and some of the results were compiled and presented to the entire facilities management department of this university in March 2016. This presentation gave an overview of the feature creation techniques and an understanding of how the buildings on their campus compare to other campuses. More discussion on the feedback from this presentation are discussed in Section \ref{sec:characterization}.