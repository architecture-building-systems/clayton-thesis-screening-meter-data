\subsection{Characterization of Building Performance Class}
\label{sec:results_benchmarking}

The second objective targeted in this study is the ability for temporal features to characterize whether a building performs well or not within it use type class. Consumption is the metric being measured, therefore its not the goal of this analysis to predict the performance of a building, its to determine which temporal characteristics are correlated with good or poor performance. This effort is related to the process of benchmarking buildings. Using the insight gained through characterization of building use type, it is possible to inform whether a building's behavior matches its peers. Once a building is part of a peer group, its necessary to understand how well that building performs within that group. In this section, the case study buildings are divided according to which percentile each fits within in its in-class performance. The buildings are divided according to percentiles, with those in the lowest 33\% are classified as “Low”, the 33 to 66\% percentile are “Intermediate”, and the top 33\% are classified as “High”. 

% The status quo of building performance benchmarking in the United States is the EnergyStar Rating system. This system relies on the Commerical Building Energy Consumption Survey that is completed every three years by the United States Department of Energy. In the United Kingdom, there is a mandatory building performance rating system requiring building owners to have Display Energy Certificates (DEC).



