\subsubsection{Long-term Pattern Consistency}
\label{sec:patternconsistency}

Breakout detection screening is a process in which each data stream is analyzed according to the tendency to shift from one performance state to another with a transition period in between. This metric is used in this context to quantify long-term pattern consistency and much of the explanation and graphic for this section are from a previous study \cite{miller_forensically_2015}. Breakout detection is a type of change point detection that determines whether a change has taken place in a time series dataset. Change detection enables the segmentation of the dataset to understand the nonstationarities caused by the underlying processes and is used in multiple disciplines involving time-series data such as quality control, navigation system monitoring, and linguistics \cite{basseville_detection_1993}. Breakout detection is applied to temporal performance data to understand general, continuous areas of performance that are similar and the transition periods between them.

In this process, an R programming package, \emph{BreakoutDetection}, is utilized, which is also developed by Twitter to process time-series data related to social media postings\footnote{https://github.com/twitter/BreakoutDetection}. This package uses statistical techniques which calculate a divergence in mean and uses robust metrics to estimate the significance of a breakout through a permutation test. The specific technical details of the breakout detection implementation can be found in a study by James et al \cite{james2014leveraging}. \emph{BreakoutDetection} uses the E-Divisive with Medians (EDM) algorithm, which is robust amongst anomalies and is able to detect multiple breakouts per time series. It is able to detect the two types of breakouts, mean shift and ramp up. Mean shift is a sudden jump in the mean of a data stream and ramp up is a gradual change of the value of a metric from one steady state to another. The algorithm has parameter settings for the minimum number of samples between breakout events that allows the user to modulate the amount of temporal detail.

The goal in using breakout detection for building performance data is to simply find when macro changes occur in sensor data stream. This detection is particularly interesting in weather insensitive data to understand when modifications are made to the underlying system in which performance is being measured. Figure \ref{fig:breakout_single} data from a single building data stream. Each color represents a group of continuous, steady-state operation and each change in color is, thus, a breakout. These breakouts could be the result of schedule or control sequence modifications, systematic behavior changes, space use type changes, etc. Creation of diversity factor schedules should target data streams which have few breakouts and the data between breakouts is the most applicable for model input. One parameter setting for breakout detection is the minimum breakout size threshold. This parameter prevents breakouts from being detected to close together, thus capturing potentially noisy behavior for the particular data set.