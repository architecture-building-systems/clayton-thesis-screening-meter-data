The creation and consolidation of measured sensor sources from the built environment and its occupants is occurring on an unprecedented scale. The Green Button Ecosystem now enables the easy extraction of data from over 60 million buildings\footnote{According to: http://www.greenbuttondata.org/}. Advanced metering infrastructure (AMI), or smart meters, have been installed on almost 52 million buildings in the US alone \cite{energy_information_administration_how_2015}. The Internet-of-Things (IoT) movement provides an array of low-cost sensors, data acquisition devices, and cloud storage. A recent study has predicted a \$70-150 billion impact of IoT in offices and \$200-350 billion in homes \cite{james_manyika_unlocking_2015}. A recent press release from the White House summarizes the impact of utilities and cities in unlocking these data \cite{_fact_2016}. It announces that 18 utilities, serving more than 2.6 million customers, will provide detailed energy data by 2017. This study also suggests that such accessibility will enable improvement of energy performance in buildings by 20\% by 2020. A vast majority of these raw data being generated are sub-hourly temporal data from meters and sensors.

Ruparathna et al. created a contemporary review of building performance analysis techniques for commercial and institutional buildings \cite{ruparathna_improving_2016}. This review was comprehensive in capturing approaches related to technical, organizational, and behavioral changes. The majority of publications considered fall within the domains of automated fault detection and diagnostics, retrofit analysis, building benchmarking, and energy auditing. These traditional techniques focus on one building or a small, related collection of buildings, such as a campus. Many require complex characteristic data about each building, such as it geometric dimensions, building materials, the age of type of mechanical systems, and other metadata, to execute the process. 

Thus, a critical issue facing the building industry is how these traditional performance analysis techniques can utilize the current explosion of detailed, temporal sources. \emph{If one has access to raw data from thousands, or even millions, of buildings, how can analysis be scaled in a reasonable way?} Someone tasked with this type of analysis needs to extract information with less known meta-data about each building and fewer inputs into the process. In response to this question, researchers from several domains have developed methods of extracting insight from raw, unlabeled data from the built environment. Often these methods fall into the category of unsupervised statistical learning. Methods from this sub-domain of machine learning are advantageous due to their ability to characterize measured or simulated performance data quickly with less analyst intervention, metadata, and ground truth labeled data. 