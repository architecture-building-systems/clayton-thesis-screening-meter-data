\section{Introduction}

The creation and consolidation of measured sensor sources from the built environment and its occupants is occurring on an unprecedented scale. The Green Button Ecosystem now enables the easy extraction of data from over 60 million buildings\footnote{According to: http://www.greenbuttondata.org/}. Advanced metering infrastructure (AMI), or smart meters, have been installed on almost 52 million buildings in the US alone \cite{energy_information_administration_how_2015}. The Internet-of-Things (IoT) movement provides an array of low-cost sensors, data acquisition devices, and cloud storage. A recent study has predicted a \$70-150 billion impact of IoT in offices and \$200-350 billion in homes \cite{james_manyika_unlocking_2015}. A recent press release from the White House summarizes the impact of utilities and cities in unlocking these data \cite{_fact_2016}. It announces that 18 utilities, serving more than 2.6 million customers, will provide detailed energy data by 2017. This study also suggests that such accessibility will enable improvement of energy performance in buildings by 20\% by 2020. A vast majority of these raw data being generated are sub-hourly temporal data from meters and sensors.

Ruparathna et al. created a contemporary review of building performance analysis techniques for commercial and institutional buildings \cite{ruparathna_improving_2016}. This review was comprehensive in capturing approaches related to technical, organizational, and behavioral changes. The majority of publications considered fall within the domains of automated fault detection and diagnostics, retrofit analysis, building benchmarking, and energy auditing. These traditional techniques focus on one building or a small, related collection of buildings, such as a campus. Many require complex characteristic data about each building, such as it geometric dimensions, building materials, the age of type of mechanical systems, and other metadata, to execute the process. 

Thus, a critical issue facing the building industry is how these traditional performance analysis techniques can utilize the current explosion of detailed, temporal sources. \emph{If one has access to raw data from hundreds, or even thousands, of buildings, how can analysis be scaled in a reasonable way?} 

\subsection{Conventional Building Analysis}
A comprehensive study of building performance tracking was completed by the California Commissioning Collaborative (CACx) and funded by the California Energy Commission (CEC) to characterize the technology, market, and research landscape in the United States. Three of the key tasks in this project focused on establishing the state of the art \cite{Effinger:2010tm}, characterizing available tools and their the barriers to adoption \cite{Ulickey:2010ut}, and establishing standard performance metrics \cite{Greensfelder:2010wl}. These reports were accomplished through investigation of the available tools and technologies on the market as well as discussions and surveys with building operators and engineers. The common theme amongst the interviews and case studies was the \emph{lack of time and expertise} on the part of the involved operations professionals. The findings showed that installation time and cost was driven by the need for an engineer to develop a full understanding of the building and systems. 