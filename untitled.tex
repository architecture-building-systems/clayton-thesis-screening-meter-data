\section{Introduction}

The creation and consolidation of measured sensor sources from the built environment and its occupants is occurring on an unprecedented scale. The Green Button Ecosystem now enables the easy extraction of data from over 60 million buildings\footnote{According to: http://www.greenbuttondata.org/}. Advanced metering infrastructure (AMI), or smart meters, have been installed on over 58.5 million buildings in the US alone\footnote{As of 2014, according to: http://www.eia.gov/tools/faqs/faq.cfm?id=108&t=3}. The Internet-of-Things (IoT) movement provides an array of low-cost sensors, data acquisition devices, and cloud storage. A recent study has predicted a \$70-150 billion impact of IoT in offices and \$200-350 billion in homes \cite{james_manyika_unlocking_2015}. A recent press release from the White House summarizes the impact of utilities and cities in unlocking these data \cite{the_white_house_fact_2016}. It announces that 18 utilities, serving more than 2.6 million customers, will provide detailed energy data by 2017. This study also suggests that such accessibility will enable improvement of energy performance in buildings by 20\% by 2020. A vast majority of these raw data being generated are sub-hourly temporal data from meters and sensors. 
%  \cite{energy_information_administration_how_2015}

Insert the data amount comparison graphic

Ruparathna et al. created a contemporary review of building performance analysis techniques for commercial and institutional buildings \cite{ruparathna_improving_2016}. This review was comprehensive in capturing approaches related to technical, organizational, and behavioral changes. The majority of publications considered fall within the domains of automated fault detection and diagnostics, retrofit analysis, building benchmarking, and energy auditing. These traditional techniques focus on one building or a small, related collection of buildings, such as a campus. Many require complex characteristic data about each building, such as it geometric dimensions, building materials, the age and type of mechanical systems, and other metadata, to execute the process. 

Another contemporary issue facing the building industry is the characterization of the commercial building stock for the purposes of benchmarking, intervention targeting, and general understanding of the way modern buildings operate. The Commerical Building Energy Consumption Survey (CBECS) is the primary means of collecting a characteristic data about the general commercial building stock in the United States. This survey is conducted every four years, the latest in 2012 in which information from over 6,700 building around the U.S. were collected for the purposes of characterization. A large amount of meta-data was collected about each building from categories such as size, vintage, geographic region, and principal activity. This data collection was done through the efforts about 250 interviewers across the country under the supervision of 17 field supervisors, 3 regional field managers, and a field director. These manpower were utilized over the course of over three years to characterize and document the commercial building stock. 

The analysis of the performance of buildings and the characterization of the building stock are important and, as discussed, quite tedious challenges in the building industry. Thus, a critical opportunity for the building industry is how these techniques can utilize the aforementioned explosion of detailed, temporal sources. \emph{If one has access to raw data from hundreds, or even thousands, of buildings, how can analysis be scaled in a robust way?} \emph{How can these data be used to inform the larger research community about the phenomenon occurring in the actual building stock?} \emph{What characteristic data about buildings can be inferred from these sources?}

