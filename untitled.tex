The creation and consolidation of measured sensor sources from the built environment and its occupants is occurring on an unprecedented scale. The Green Button Ecosystem now enables the easy extraction of data from over 60 million buildings\footnote{http://www.greenbuttondata.org/}. Advanced metering infrastructure (AMI), or smart meters, have been installed on almost 52 million buildings in the US alone \cite{energy_information_administration_how_2015}. The Internet-of-Things (IoT) movement provides an array of low-cost sensors, data acquisition devices, and cloud storage. A recent study has predicted a \$70-150 billion impact of IoT in offices and \$200-350 billion in homes \cite{james_manyika_unlocking_2015}. A recent press release from the White House summarizes the impact of utilities and cities in unlocking these data \cite{_fact_2016}. It announces that 18 utilities, serving more than 2.6 million customers, will provide detailed energy data by 2017. This study also suggests that such accessibility will enable improvement of energy performance in buildings by 20\% by 2020. A vast majority of these raw data being generated are sub-hourly temporal data from meters and sensors.