\subsubsection{University Dormitory and Laboratory Comparison}
\label{sec:dormvslab}

The random forest classification model and variable importance metrics provide an indication of how the features characterize the a building's use. A deeper investigation of the features with a comparison between two use types is useful to understand the characterization potential of various subsets of features. For this example, two building type classifications are compared that showed strong distinction from each other in the random forest model: university laboratories and dormitories. For this comparison, the highly comparative time-series analysis (hctsa) code repository is used as a toolkit for analysis of the generated temporal features in this study \cite{Fulcher_2013}. This toolkit has various visualization tools that enable analysis of the predictive capabilities of temporal features. This toolkit includes a library of temporal features, however at this point in the analysis, only the features developed in this study are used Figure \ref{fig:featurecluserting_dormvslabs} shows the top forty features in differentiating university laboratories and dormitories using a simple linear classifier model. These features are clustered according to their absolute correlation coefficients in order to understand how many unique sets of informative features are present. Groups of features in the same cluster are essentially giving the same type of information about the differences between a certain set of tested classes. In the case of laboratories and dormitories, there are eight sets of clusters giving information about this distinction. The first cluster is a pair of breakout detection