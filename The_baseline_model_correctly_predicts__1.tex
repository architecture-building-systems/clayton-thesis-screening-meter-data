The baseline model correctly predicts the labels with a 18.1\% accuracy, while the features influenced models were 38.5\% for Non-Temporal, 45.3\% for the Temporal and 45.7\% for the combined model.  The baseline model represents common practice in which a class is chosen based on the probability distribution of that class occurring in the labeled dataset. The combined feature set more than doubles the probability of predicting this piece of meta-data.

Mean accuracy of multi-label classification models as done in this analysis is a harsh metric as it forces the model to make a single choice for labeling each sample. In practice, we would not want a model that completely makes this decision; we would simply want the model to inform us what the probability that a sample fits within a class. For example, there could be 45\% chance an unlabeled account is an office, a 35\% chance it is a school and 20\% chance it is a grocery store. The reason we choose mean model accuracy in this report it to communicate a simplified message of the techniques and the progress we’ve made thus far. The fact that the overall classification model accuracy is around 40-60\% for a classification model with ten classes is not discouraging. It is the improvement in mean accuracy from baselines that is the focus and this has been demonstrated so far in the project. Additionally, there are other classification metrics including precision, recall, and F-Score that we have left out from this report that can be used to determine model usefulness. 

We can also see in detail how the model predicts the classes for each by creating and analyzing a classification confusion matrix. Figure \ref{fig:industry_classification} illustrates this matrix for the combined model. We can observe that two of the largest classes, Retail and Finance, have the highest accuracy rates at over 55-60\% with several other categories being misclassified within them.  This issue is common with imbalanced classification models and further feature development would improve the model by better characterizing the difference between each class.
