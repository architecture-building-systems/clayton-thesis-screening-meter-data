\subsection{Smart Meter Analytics}
\label{SmartMeter}
Advanced Metering Infrastructure (AMI), also known as smart meter systems, is a network of energy meters, most often focused on the electrical power measurement of a whole building. These systems are implemented and utilized by electrical utility providers. Conventional metering infrastructure only facilitates monthly data collection for billing purposes, while the new AMI framework allows for sub-hourly electrical demand readings. These data are primarily used for demand characterization and billing, however, many additional uses are being discovered. A wide-range of studies have been completed in recent years to focus on a range of issues related to automatically extracting information from these data using unsupervised techniques. In this section, three sub-categories of application are discussed: load profiling, account classification, and disaggregation.

\subsubsection{Load Profiling}
Load profiling is the process of grouping temporal subsequences of measured energy data for the purpose of characterizing the typical behavior of an individual customer. It involves time-series clustering and feature extraction. Chicco et al. \cite{chicco_support_2009} provides an original example in our review of this process using support vector machine clustering. Gullo et al. \cite{gullo_low-voltage_2009} and R\"as\"anen et al. \cite{rasanen_feature-based_2009} took the process further by introducing a framework of various clustering procedures that were implemented on case studies. Ramos et al. \cite{ramos_typical_2012}, Iglesias et al. \cite{iglesias_analysis_2013} and Panapakidis et al. \cite{panapakidis_evaluation_2015} tested various conventional and new clustering methods and similarity metrics in order to determine those most applicable to electrical load profiling. Chicco et al. \cite{chicco_electrical_2013} explored new clustering techniques based on ant colony grouping while Pan et al. \cite{pan_kernel-based_2015} discovered the use of kernal PCA for the same purpose. Several groups of researchers such as Lavin and Klabjan \cite{lavin_clustering_2014} and Green et al. \cite{green_divide_2014} have found effective use in using the basic K-Means clustering algorithm for load profiling. Shahzadeh et al. \cite{shahzadeh_improving_2015} discussed the use of profiling as applied to forecast accuracy of temporal data. Two studies diverge from the standard profile development using clustering paradigm. The first is by De Silva et al. \cite{de_silva_data_2011} who uses Incremental Summarization and Pattern Characterization (ISPC) instead of clustering to find load profiles. The other is the visual analytics-based approach of creating a smart meter analytics dashboard by Nezhad et al. \cite{jarrah_nezhad_smartd:_2014} to create and inspect typical load profiles.

\subsubsection{Customer Classification}
Automated account classification is the next sub-category that utilizes unsupervised learning technques within the smart meter domain. These methods often employ load profile clustering as a first step but differentiate themselves in using those features to classify accounts, or buildings, that fit within various categories. Therefore, account classification is a type of manual semi-supervised analysis utilizing load profiling as a basis. The study by Figueiredo et al. \cite{figueiredo_electric_2005} harnessed K-Means and a labeled sample from accounts in Portugal to showcase this concept. Verdu et al. \cite{verdu_classification_2006} and R\"as\"anen et al. \cite{rasanen_reducing_2008} applied self-organizing maps (SOM) to accomplish a similar study that classifies accounts according to applicability of several demand response scenarios. Vale et al. \cite{vale_data_2009} gives an overview of a general data mining framework focused on characterizing customers. Florita et al. \cite{florita_classification_2012} diverge from the use of measured data by creating a massive amount of simulation data of load profiles to quantify energy storage applications for the power grid. Fagiani et al. \cite{fagiani_novelty_2015} use Markov Model novelty detection to automatically classify customers who potentially have leakage or waste issues. \c Cakmak et al. \cite{cakmak_new_2014} and Liu et al. \cite{liu_smas:_2015} test new visual analytics techniques within more holistic analysis framework for analyzing customers. Borgeson used various clustering and occupancy detection techniques to analyze a large AMI data set from California \cite{borgeson_targeted_2013}. Bidoki et al. tested various clustering techniques to evaluate applicability for customer classification \cite{bidoki_evaluating_2010}. A recent study in Korea develops a new clustering technique in segmenting customers to analyze demand response incentives \cite{jang_variability_2016}.

\subsubsection{Disaggregation}
The last area of smart meter data analysis is the field of meter disaggregation. Disaggregation attempts decompose a measurement signal from a high level reading to the individual loads being measured. This domain is well-researched from a supervised model perspective but recent attempts at unsupervised, pattern-based disaggregation were developed to facilitate implementation on unlabeled smart meter data. Shao et al. \cite{shao_temporal_2013} use Dirichlet Process Gaussian Mixture Models to find and disaggregate patterns in sub-hourly meter data. Reinhardt and Koessler \cite{reinhardt_powersax:_2014} use a version of symbolic aggregate approximation (SAX) to extract and identify disaggregated patterns for the purpose of prediction. These studies are also unique in that few of the disaggregation studies focus on commercial buildings as opposed to residential buildings.

