\subsubsection{Correlation Coefficient}
\label{sec:weathercorrelationcoeff}

Data stream influence characterization is the process of roughly classifying the dataset into streams and subsequences based on weather conditions sensitivity. A feature is developed in a study of evaluation of campus data for simulation feedback and the following is a summarization of this technique \cite{miller_forensically_2015}. This evaluation is important in understanding what measured performance is due to heating, cooling, and ventilation systems (HVAC) responses to outdoor conditions and what is due to schedule, occupancy, lighting, and miscellaneous loading conditions which are weather independent. Performance data that is influenced by weather can be used to better understand the HVAC system operation or be weather-normalized to understand occupant diversity schedules. %Non-weather sensitive data streams are used with less pre-processing to create diversity schedules and to calculate miscellaneous and lighting load power densities. T

The Spearman Rank Order Correlation (ROC) is used to evaluate the positive or negative correlation between each performance measurement stream and the outdoor air dry bulb temperature. This technique has been previously used for weather sensitivity analysis \cite{coughlin_statistical_2009}. The ROC coefficient, $\rho$, is calculated according to a comparison of two data streams, $X$ and $Y$, in which the values at each time step, $X_i$ and $Y_i$, are converted to a relative rank of magnitude, $x_i$ and $y_i$, according to its respective dataset. These rankings are then used to calculate $\rho$ that varies between +1 and -1 with each extreme corresponding to a perfect positive and negative correlation respectively. A value of 0 signifies no correlation between the datasets. This $\rho$ value for a time-series is calculated according to Equation \ref{eq:spearman}.

\begin{equation}
\rho = 1 - \frac{6\sum d_i^2}{n(n^2-1)}
\label{eq:spearman}
\end{equation}

The difference between the data stream rankings, $x_i$ and $y_i$, is signified by a difference value, $d_i$, and the number of samples compared in each dataset is signified by $n$. Figure \ref{fig:weather_examples_plot} illustrates the calculation of the ROC coefficient, $\rho$ for three examples. The cooling sensitive data set shows a strong positive correlation between outside air temperature and energy consumption with a $\rho$ value of 0.934. As the outside air temperature increases, the power consumption measured by this meter increases. The heating sensitive dataset shown has a strong negative correlation with a $\rho$ of -0.68. A weather insensitive dataset is shown in the middle which has a $\rho$ of 0.0, signifying no weather correlation, which is obvious due to the four levels of consumption which are independent from outdoor air conditions.