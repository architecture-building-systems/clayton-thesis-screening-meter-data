\subsection{Load shape regression-based Features}
\label{sec:regressionmetrics}

Prediction of electrical loads based on their shape and trends over time is a mature field developed to forecast consumption to detect anomalies and analyze the impact of demand response and efficiency measures. The most common technique in this category is the use of heating and cooling degree days to normalize monthly consumption \cite{fels_prism:_1986}. Over the years, various other techniques have been developed using techniques such as neural networks, ARIMA models, and more complex regression \cite{taylor_comparison_2006}. However, simplified techniques have retained their usefulness over time due to ease of implementation and accuracy. In the context of temporal feature creation, a regression model provides various metrics that describe how well a meter conforms to expected assumptions. For example, if actual measurements and predicted consumption match well, the underlying behavior of a energy-consuming systems in the building has been captured adequately. If not, there is uncharacterized phenomenon that will need to be captured with a different type of model or feature. 

A contemporary, simplified load prediction technique is selected to create temporal features that capture whether electrical measurement is simply a function of time-of-week scheduling. This model was developed by Matthieu et al. and Price and implemented mostly in the context of electrical demand response evaluation \cite{price_methods_2010, mathieu_quantifying_2011}. The premise of the model is based on two features: a time-of-week indicator