\subsubsection{Discussion with Campus Case Study Subjects}
\label{sec:performanceclass_oncasestudy}

In a situation similar as the discussion about building use type, participants in the case studies were guided through the process of analysis using a subset of features from buildings on their campus. Figure \ref{fig:breakdown_performanceclass} illustrates a graphic that was shown to the groups. In this case, the buildings are divided into two classes: \emph{Good} and \emph{Bad}. These classes are based on whether the building falls in the upper or lower 50\% within its class. The first observation by the case study participants is that the load diversity, or the daily maximum versus minimum, is a strong indicator of the performance class. This fact is not surprising as this metric indicates the magnitude of the base load consumption as compared to the peak. Other relatively strong differentiators in this case is cooling energy, seasonal changes, and weekly specificity.