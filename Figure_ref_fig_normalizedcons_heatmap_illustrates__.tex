Figure \ref{fig:normalizedcons_heatmap} illustrates the same normalized consumption metric as applied to all of case study buildings. There are five segments of buildings based on the primary use types within the set: offices, university laboratories, university classrooms, primary/secondary schools, and university dormitories. These metrics are visualized in this way in order to understand the difference between each of these use types for each of the presented metrics. Each row of the heatmap for each segment is the values of the feature for a single building, while the x-axis is the time range for all buildings. Not all of the case study buildings have a January to December time range. For these cases, the data was rearranged so that a continuous set of January to December data is available to be visualized in the heatmap. The aggregation metrics themselves are not calculated with this rearranged vector, it is only for visualization purposes. Like Figure \ref{fig:normalizedmag}, this type of graphic is used to visualize many of the temporal features in this Section. From this metric in particular, one will notice that university labs have a systematically higher consumption over time as compared to the other use types. One will also see the dark vertical lines across the time range indicating weekend consumption as compared to weekday. This particular pattern is absent from university dormitories due to their more continuous energy consumption.